\documentclass[12pt, a4paper]{article}
\usepackage{amsmath}
\usepackage{amsfonts}
\usepackage{amsthm}
\usepackage{mathtools}
\newtheorem{theorem}{Theorem}[section]
\newtheorem{definition}{Definition}[section]
\numberwithin{equation}{section}
\usepackage{pgfplots}
\pgfplotsset{width=10cm,compat=1.9}
\graphicspath{ {img/} }
\DeclareGraphicsExtensions{.png}

\title{Intersection between ellipse and line segment}
\author{Kristian Wichmann}

\begin{document}
\maketitle

\section{Equation for the ellipse}
Let the semimajor and semininor axis of the ellipse be given by $a$ and $b$. Assuming these to lie along the $x$ and $y$ axis of the coordinate system, the standard equation is:
\begin{equation}
\label{ellipse_equation}
\left(\frac{x}{a}\right)^2+\left(\frac{y}{b}\right)^2=1
\end{equation}

\section{Equation for the line segment}
Let the line segment run bewteen $P(p_1,p_2)$ and $Q(q_1,q_2)$. A directional vector for the segment is:
\begin{equation}
\label{line_equation}
\vec{PQ}=
\begin{pmatrix}
q_1-p_1 \\
q_2-p_2
\end{pmatrix}
\end{equation}
Hence, a parametric equation for the line segment is:
\begin{equation}
\begin{pmatrix}
x \\
y
\end{pmatrix}
=
\begin{pmatrix}
p_1 \\
p_2
\end{pmatrix}
+t
\begin{pmatrix}
q_1-p_1 \\
q_2-p_2
\end{pmatrix}
=
\begin{pmatrix}
p_1 + t(q_1-p_1) \\
p_2 + t(q_2-p_2)
\end{pmatrix}
,\quad t\in[0,1]
\end{equation}

\section{Solving the equations}
We now search for values of $t$ which solve both \ref{ellipse_equation} and \ref{line_equation}. According to \ref{line_equation} we must have:
\begin{equation}
p_1 + t(q_1-p_1)
\end{equation}
The square is:
\begin{equation}
x^2=p_1^2+t^2(q_1-p_1)^2+2p_1t(q_1-p_1)
\end{equation}
Ordering by powers of $t$ this is:
\begin{equation}
\label{x_squared}
x^2=(q_1-p_1)^2t^2+2p_1(q_1-p_1)t+p_1^2
\end{equation}
By analogy, for $y$ we get:
\begin{equation}
\label{y_squared}
y^2=(q_2-p_2)^2t^2+2p_2(q_2-p_2)t+p_2^2
\end{equation}
Now insert \ref{x_squared} and \ref{y_squared} into \ref{ellipse_equation}:
\begin{equation}
\frac{(q_1-p_1)^2}{a^2}t^2+\frac{2p_1(q_1-p_1)}{a^2}t+\frac{p_1^2}{a^2}+\frac{(q_2-p_2)^2}{b^2}t^2+\frac{2p_2(q_2-p_2)}{b^2}t+\frac{p_2^2}{b^2}=1
\end{equation}
Collect terms according to power of $t$:
\begin{align}
\underbrace{\left(\frac{(q_1-p_1)^2}{a^2}+\frac{(q_2-p_2)^2}{b^2}\right)}_{c_2}t^2 &\\
+\underbrace{\left(\frac{2p_1(q_1-p_1)}{a^2}+\frac{2p_2(q_2-p_2)}{b^2}\right)}_{c_1}t &\\
+\underbrace{\left(\frac{p_1^2}{a^2}+\frac{p_2^2}{b^2}-1\right)}_{c_0} & = 0
\end{align}
This is a quadratic equation in $t$. The corresponding discriminant is:
\begin{equation}
d=c_1^2-4c_0c_2
\end{equation}
If $d<0$ there's no solution. If $d>0$, there's two:
\begin{equation}
t=\frac{-c_1\pm\sqrt{d}}{2c_2}
\end{equation}
If $d$ is exactly zero, there's one solution $t=-\frac{c_1}{2c_2}$.

\section{Checking the solutions}
Remember that the line segment only stretches from $t=0$ to $t=1$. So it must be checked that the solutions (if any) do indeed lie in the interval $[0,1]$. Otherwise, there is no intersection (though an extension of the line segment does intersect the ellipse).

\end{document}